\chapter{Conclusion}
\label{sec:conclusion}
\chaptermark{Conclusion}

Visual processing of objects makes use of both feedforward and feedback streams of information. In this thesis, we showed how recurrent neural networks that make use of grouping mechanisms can perform a wide variety of tasks, including contour integration, figure-ground segregation (both in artificial and natural scenes), grouping of 3D surfaces, and visual saliency prediction. A key feature of our models is the use of grouping neurons whose activity represents tentative objects (“proto-objects”) based on the integration of local feature information. Grouping neurons receive input from an organized set of local feature neurons, and project modulatory feedback to those same neurons. Additionally, inhibition at both the local feature level and the object representation level biases the interpretation
of the visual scene in agreement with principles from Gestalt psychology. Our models can explains several sets of psychophysical and neurophysiological results~\citep{He_Nakayama95,Zhou_etal00,Qiu_etal07,Chen_etal14,Williford_vonderHeydt16}, and makes testable predictions about the influence of neuronal feedback and attentional selection on neural responses across different visual areas. Our proposed models also provide a framework for understanding how object-based attention is able to select both objects and the features associated with them.

Future work will need to further explore how grouping mechanisms operate in 3D. Understanding how the brain efficiently represents 3D surfaces, possibly with the combination of basis functions and grouping mechanisms, will be critical for advancing our understanding of the cortical mechanisms of perceptual organization. Currently, our models can only handle static images. Additional work on how grouping is performed on other feature types, such as motion, and how information from different features is ultimately integrated into a whole, will be important for extending our models to handle the rich spatiotemporal information present in videos. Finally, the interaction between mid-level visual representations of proto-objects and higher-level visual representations of object identity is still unknown. Understanding how object identity influences figure-ground segmentation and other grouping processes is a promising area of future research.