%% FRONTMATTER
\begin{frontmatter}

% generate title
\maketitle

\begin{abstract}
% bh taken from NRSA/thesis proposal specific aims
% Chapter 1: Introduction
% Chapter 2: Contour Integration (Chen et al., '14) and Border Ownership
% Chapter 3: Figure-Ground Segmentation in Natural Images
% Chapter 4: 3D Surface Representation using Basis Functions
% Chapter 5: 3D Proto-object Based Saliency
  
The visual brain faces the difficult task of reconstructing a three-dimensional (3D) world from two-dimensional (2D) retinal images. In doing so, visual information is organized in terms of objects in 3D space, and this organization is the basis for selective attention, object recognition, and action planning. In complex visual scenes, both the foreground and the background are rich in features of different types, scales, \etc The brain must find a way to group together the features that belong to objects on the foreground, and distinguish them from features in the background.

The goal of this thesis is to understand how the neural circuits in primate cortex accomplish this task using feedback grouping mechanisms for object-based vision and attention. In Chapter 1, we introduce the background information needed to understand the physiology and  previous modeling experiments. In Chapter 2, we propose a quantitative neural model of contour grouping constrained by recent physiological data. We validate the model by reproducing several experimental results, including the measure of contour-response $d'$, as well as the magnitude and time course of neuronal responses to contours. In Chapter 3, we extend this model to natural images, and the results are quantitatively compared with human-generated segmentations and figure-ground labels (Berkeley Segmentation Dataset). Beginning with Chapter 4, we shift our focus to the representation of 3D information in the visual system. First, we show that 3D surfaces can be represented by a feedforward, linear combination of basis functions whose response properties are similar to those of disparity-selective neurons commonly found in early visual cortex. With our model, we are able to reproduce results from a set of psychophysical experiments where attention has to be directed to surfaces. In Chapter 5, we propose a model of 3D visual saliency and show that the added depth information improves saliency prediction. Overall, this work will investigate whether feedback grouping mechanisms are fundamental for linking early feature representations to perceptual objects. The models developed will address how visual features are grouped into 2D and 3D object representations.

\vspace{1cm}

\noindent Primary Reader: Ernst Niebur\\
Secondary Reader: R{\"u}diger von der Heydt

\end{abstract}

\begin{acknowledgment}

As I was sitting on the boat, I was thinking about how the trip was really a metaphor for my PhD experience. I have had opportunities to steer the boat (sometimes in the wrong direction!), gone through both good and bad weather, and through it all, have had a good captain whom I can trust to guide me to the finish.

\end{acknowledgment}

\begin{dedication}
 
This thesis is dedicated to my younger sister Joy, whose perseverance in the face of adversity has been my inspiration.

\end{dedication}

% generate table of contents
\tableofcontents

% generate list of tables
\listoftables

% generate list of figures
\listoffigures

\end{frontmatter}

%%% Local Variables:
%%% mode: latex
%%% TeX-master: "root"
%%% End:
