%% FRONTMATTER
\begin{frontmatter}

% generate title
\maketitle

\begin{abstract}
% bh taken from NRSA/thesis proposal specific aims
% Chapter 1: Introduction
% Chapter 2: Contour Integration/Border Ownership
% Chapter 3: Figure-Ground Organization in Natural Scenes
% Chapter 4: 3D Surface Representation (Basis Functions)
% Chapter 5: 3D Proto-object Based Saliency
  
The visual brain faces the difficult task of reconstructing a three-dimensional (3D) world from two-dimensional (2D) images projected onto the two retinae. In doing so, visual information is organized in terms of objects in 3D space, and this organization is the basis for visual perception. In complex visual scenes, both the foreground and the background are rich in features of different types. The brain must find a way to group together the features that belong to objects on the foreground and distinguish them from features in the background.

The goal of this thesis is to understand how the neural circuits in primate cortex accomplish this task using grouping mechanisms for object-based vision and attention. Through computational modeling, I show that grouping mechanisms are fundamental for linking early feature representations to tentative perceptual objects known as proto-objects. Previous models on the neural coding of border ownership have identified a plausible network architecture for proto-object based perceptual organization. I extend these models to explain how the same grouping framework can be used to perform contour integration, border-ownership assignment, grouping of 3D surfaces, and 3D visual saliency. My models offer several falsifiable predictions which can be tested in future experiments. My models also clarify how top-down attention interfaces with the neural circuits responsible for grouping together the features of an object. Overall, the models developed address the important question of how visual features are grouped into 2D and 3D object representations.

\vspace{1cm}

\noindent Primary Reader: Ernst Niebur\\
Secondary Reader: R{\"u}diger von der Heydt

\end{abstract}

\begin{acknowledgment}

I had the privilege of sailing with Ernst to the neuroscience retreat one year. As I was sitting on the boat, I thought about how the trip was really a metaphor for my PhD. I have had opportunities to steer the boat (sometimes in the wrong direction!) and I have gone through both good and bad weather. Through it all, I am thankful to have had Ernst as my captain who guided me to the finish.

I would like to thank my family for their continued love and support. I thank my wife, Mingming, and my daughter, Katherine, for bearing with me in these final months. I thank my parents, Benjamin and Jennifer, for all the sacrifices they made for me and my siblings. I thank my brother Blair for his moral support and my sister Joy for being an inspiration to me.

I would also like to thank my thesis committee members, R{\"u}diger and Kechen, for their help and guidance during my PhD. Finally, I would like to thank my friend Matthew, for all the lunches and discussions that we shared together, and my labmates Danny and Grant, who made each day in lab an interesting one.

\textbf{Soli Deo Gloria}

\vspace{-1cm} % remove extra white space to keep on one page

\end{acknowledgment}

\begin{dedication}
 
This thesis is dedicated to Joy, who first got me interested in the study of vision. Your perseverance in the face of adversity has been my inspiration.

\end{dedication}

% generate table of contents
\tableofcontents

% generate list of tables
\listoftables

% generate list of figures
\listoffigures

\end{frontmatter}

%%% Local Variables:
%%% mode: latex
%%% TeX-master: "root"
%%% End:
